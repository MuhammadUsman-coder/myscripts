\begin{abstract}
The main goal of this article is to present the fundamental theoretical concepts for the tutorial presented in IEEE NetSoft 2020. The article explores the use of software in the 5G system composed of the Radio Access Network (RAN) and the core components, following the standards defined by 3GPP, particularly the Release 15. The article provides a brief overview of mobile cellular networks, including basic concepts, operations, and evolution through the called `generations' of mobile networks. From a software perspective, RAN is presented in the context of 4G and 5G networks, which includes the virtualization and disaggregation concepts. A significant part of the article is dedicated to 5G networks and beyond, focusing on core, \textit{i.e.}, considering the Service-Based Architecture (SBA), due to its relevance and totally softwarized approach. Finally, the article briefly describes the demonstrations presented in IEEE NetSoft 2020, providing the link for the repository that has all material employed in the tutorial.
\end{abstract}