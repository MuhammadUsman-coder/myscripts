\section{Final considerations}\label{sec:conclusao}

The academy has already started to investigate potential challenges for the next generation of wireless mobile networks. Naturally, there is still a lot of uncertainty, and several forecasts cannot be consolidated. However, some themes are beginning to be addressed preliminarily in the 3GPP release scope. For example, Releases 16 and 17 improvements in URLLC and the use of SON to make the 5G networks more efficient are expected, indicated that forecasts such as MBRLLC and mURLLC, as well as, the broad adoption of Machine Learning and Artificial Intelligence, can become a reality in the future. In the following, we briefly discuss some topics addressed in the next two releases, scheduled for June 2020 (Release 16) and September 2021 (Release 17).


\subsection*{Data-driven network}

The introduction of NWDAF in Release 15 is an essential step for adopting ML and AL in a 5G network, but it is only the beginning. The complete specification of the data collection and analysis framework must be completed in Release 16. From that point, solutions based on ML and AI can use the information collected by NWDAF to perform tasks such as predicting and mobility optimization, detecting anomalies, QoS predicting, and data correction. In this context, some of the goals for Release 17 are:

\begin{itemize}
    \item Predictable network performance assisted by NWDAF;
    \item UE oriented data analysis;
    \item Expose of NWDAF data analysis for user applications;
    \item NWDAF supporting the detection of anomalous events and helping to analyze their causes.
\end{itemize}

In the long term, the objective is to use ML and AI techniques to automate network management with the minimum possible human intervention. It is fundamental to highlight that this management should involve different network types (\textit{i.e.}, heterogeneous networks) connecting to a common core based on the SBA model.

\subsection*{Improvements for vertical domains}

5G networks have been designed with particular attention to vertical domains (usually called only as verticals), \textit{i.e.}, specific sectors or groups of companies in which similar products and services are developed, produced, and supplied. In this context, there are several directions and contributions in Release 16, for example:

\begin{itemize}
    \item Support to Time Sensitive Communication (TSC);
    \item Non-Public Networks (NPNs), \textit{i.e.}, private networks;
    \item Support to 5G LAN services;
    \item Advanced location services.
\end{itemize}

In Release 17, improvements in support for NPNs and Proximity Services (ProSe) are planned, which are also useful in public security scenarios.

\subsection*{Security evolution}

Security has always been a concern in mobile wireless networks, especially from 3G with the wide use of the infrastructure for data transport and its natural integration with the Internet. Therefore, the security issue in 5G networks has been addressed from its design. It has been a recurring topic, mainly due to the strategic importance of 5G in the economy and social issues. There is a list of security contributions planned for Release 16, of which some relevant items are highlighted the following:

\begin{itemize}
    \item Support for NPNs with new authentication schemes;
    \item Network slice specific authentication option, in addition to primary authentication;
    \item Advanced security for Radio Resource Control and NAS signaling;
    \item Support for integrity protection in the user plane, covering the three 5G network scenarios, \textit{i.e.}, eMBB, URLLC, and mMTC.
\end{itemize}

\subsection*{Other advances}

There are several other contributions in Release 16 and 17 that introduce new solutions and offer opportunities for the research. This opportunity occurs because specific solutions are not entirely defined due to factors such as complexity of the problem addressed or inability to identify in the standard the specifics that the answer will have to deal with that problem. In the following, we list some topics of these two Releases (16 and 17) that, although they have already been investigated, still present open questions, especially in a practical context and with the potential large-scale application. 

\begin{itemize}
    \item Advances on Vehicle-to-everything (V2X) -- platoon formation, autonomous steering, and remote steering;
    \item Access to unlicensed spectrum using 5G New Radio; 
    \item Dynamic Spectrum Sharing (DSS);
    \item IoT for Non-Terrestrial Networks (NTNs);
    \item Support for Unmanned Aerial Systems (UAS).
\end{itemize}

Finally, we can see the increasingly strong integration between the areas of Telecommunications and Information Technology. On the one hand, the adoption of software and cloud-native model in 5G systems, \textit{i.e.}, RAN and core has grown. On the other hand, the interest to create software solutions that meet the specific needs of mobile wireless networks, such as edge computing and signal processing systems, has also increased. This integration will still be explored mainly in the next years, to make development even more agile and interoperable. An example of this change in view is the distribution (from Release 15) of 3GPP specifications in the format of readable data serialization (YAML - \textit{YAML Ain't Markup Language})~\cite{3GPP:OpenAPI}, allowing the design and development of standardized interfaces and with high flexibility. Moreover, the use of cloud-native technologies will allow extremely automated management and control, with minimal (or even without) human intervention, which has been called Zero Touch Network $\&$ Service Management (ZSM)~\cite{etsi:ZSM}. These developments are expected to change the business model of telecommunication operators, allowing them to expand Business to Consumer (B2C) and Business to Business (B2B) relationships.